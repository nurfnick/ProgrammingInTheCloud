 % $Header: /cvsroot/latex-beamer/latex-beamer/solutions/conference-talks/conference-ornate-20min.en.tex,v 1.7 2007/01/28 20:48:23 tantau Exp $
\documentclass[xcolor=pdftex,dvipsnames,table]{beamer}
%\documentclass[amscd,amssymb,11pt]{beamer}

\usepackage{xfrac}
\usepackage{float}
\usepackage{pgf,tikz}
\usetikzlibrary{trees}
\usetikzlibrary{arrows}
\usepackage{color}
\usepackage{hyperref}
\usepackage{multirow}
\usepackage{graphics}
\usepackage{graphicx}
\usepackage{float}
\usepackage{multicol}
\usepackage{soul}


\DeclareGraphicsRule{.tif}{png}{.png}{`convert #1 `basename #1 .tif`.png}

% This file is a solution template for:

% - Talk at a conference/colloquium.
% - Talk length is about 20min.
% - Style is ornate.

\mode<presentation>
{
  \usetheme{Copenhagen}
\usecolortheme[RGB={255,82,0}]{structure}
%\usetheme{Madrid}
}

\usepackage[all]{xy}



\usepackage{color}
\usepackage[english]{babel}
% or whatever

\usepackage[latin1]{inputenc}
% or whatever

\usepackage{hyperref}
\hypersetup{pdfnewwindow=true, pdffitwindow=true}

\title{Programming in the Cloud}
\author{Nicholas C. Jacob}
\date{\today}




\begin{document}

\begin{frame}
\maketitle
%  \titlepage
\begin{abstract}
Utilizing free, open source resources, it is easier than ever to integrate programming into your courses.  We will examine coding in python and R using cloud computing.  We will show how students can create reports and share them with class or the world with just an internet connection.
\end{abstract}
\end{frame}


\begin{frame}{Git}
Git can be very intimidating! \pause  
Uses command line interface.  \pause

Git has its own language.  
\begin{itemize}
\item Push (Save your changes)\pause 
\item Pull (See if anyone else made changes)\pause
\item Commit (Compare the original file to your changes and approve)\pause
\item  Merge (Accept other\rq{}s commits) \pause
\item Revert (Return to a previous version of your code)\pause
\end{itemize}


\end{frame}

\begin{frame}{GitHub}{Git}

GitHub Advantages.
\begin{itemize}
\item Web based interface \pause
\item Makes collaboration easy 
\item Specifically built for collaboration\pause
\item Others can help debug\pause
\item \href{https://github.com/nurfnick/Data_Sets_For_Stats/blob/master/rExamples/hockey_paper.Rmd} {Code is shareable and stored on web}\pause 
\item Free and used by many companies\pause
\item \href{https://nurfnick.github.io/Data_Sets_For_Stats/rExamples/hockey_paper.html}{Non-dynamic html can be shared from the site} \pause
\item \href{https://github.com/nurfnick}{Has a social aspect} \pause
\item \href{https://github.com/nurfnick/Data_Sets_For_Stats}{Easier to update than LMS shells, students can make commits too}
\end{itemize}
\end{frame}


\begin{frame}{Markdown (.md)}{Markdown}
\begin{itemize}
\item html ready coding \pause
\item Headers get \# \pause
\item Bold, italics, etc. Done with *, ** \pause
\item Numbering can be done with 1. 2. 3. \pause
\item Supports \LaTeX math modes \$ and \$\$ \pause
\item \href{https://www.tablesgenerator.com/markdown_tables}{Tables} \pause
\item Links [Text](hyperlink) \pause
\item \href{https://www.markdownguide.org/cheat-sheet/}{And More}
\end{itemize}
\end{frame}

\begin{frame}{Python (.py)}{Python}
\begin{itemize}
\item Python is used across BIG\pause
\item Code is readable\pause
\item Libraries are extensive\pause
\item Approachable for beginners
\end{itemize}
\end{frame}

\begin{frame}{Jupyter Notebooks (.ipynb)}{Python}
\begin{itemize}
\item Supports python code  \pause
\item Supports regular text \pause
\item Supports markdown (\LaTeX lite) \pause
\item \href{https://colab.research.google.com/github/nurfnick/Numerical_Methods/blob/master/ProjectPart3.ipynb}{Can create reports with embedded code}
\end{itemize}
\end{frame}
\begin{frame}{Colab}{Python}
\begin{itemize}
\item Runs Jupyter notebooks \pause
\item Requires gmail account \pause
\item Upload, download, open github with link  \pause
\item Runs faster than my office desktop  \pause
\item Shareable inside environment or \pause
\item Save to github (no git language required)  \pause
\item Most packages pre-loaded, can load others through command line commands  \pause
\item \href{https://colab.research.google.com/notebooks/intro.ipynb}{Colab\rq{}s Intro Page}
\end{itemize}

\end{frame}

\begin{frame}{R}

\begin{itemize}
\item Widely used statistical tool \pause
\item Large community, lots of packages (and MAA training for integration into the intro to stat curriculum statprep)\pause
\item Command line based
\end{itemize}
\end{frame}

\begin{frame}{RStudio}{R}

\begin{itemize}
\item Suite of packages for using R (library packages, console, git, terminal, stored data and functions) \pause
\item Saved as projects \pause
\item Considerable set up for local machines  \pause
\item Can set up server for campus \pause
\item Create reports using Rmd and the markdown language  \pause
\item Knit documents into html, pdf or word \pause
\end{itemize}

\end{frame}
\begin{frame}{RStudio.cloud}{R}

\begin{itemize}
\item RStudio but web-based \pause
\item Lots of limitations on storage and ram in free version \pause
\item Can create project for entire class  \pause
\item \href{https://rpubs.com/nurfnick/696714}{Knit html can be published to the web freely}
\end{itemize}

\end{frame}
\begin{frame}{Take Aways}
\begin{itemize}
\item Get Students coding as quickly as possible
\item Students will be on the struggle bus no matter what you do
\item Create environments for them to share
\item Help Students Create a Portfolio of Work
\item Have lots of discussions about backing up work in the cloud environments
\end{itemize}
\end{frame}
\end{document}
